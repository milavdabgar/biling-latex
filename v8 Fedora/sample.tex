\documentclass[]{article}
\usepackage{fontspec}
\usepackage{polyglossia}
\usepackage{xcolor}
\usepackage{listings}
\usepackage{fancyvrb}
\usepackage{longtable,booktabs,array}
\usepackage{graphicx}
\usepackage{hyperref}
\usepackage{url}
\usepackage{xltxtra,xunicode}

\setmainlanguage{english}
\setotherlanguage{gujarati}

% Set fonts
\setmainfont{Noto Sans}
\newfontfamily\gujaratifont{Noto Sans Gujarati}[Script=Gujarati]

% Configure listings for code highlighting
\lstset{
    basicstyle=\ttfamily\small,
    breaklines=true,
    columns=flexible,
    keepspaces=true,
    numbers=left,
    numberstyle=\tiny\color{gray},
    showstringspaces=false,
    frame=single,
    framesep=3pt
}

% Define tightlist command
\providecommand{\tightlist}{%
  \setlength{\itemsep}{0pt}\setlength{\parskip}{0pt}}


% Title, authors, and date

\begin{document}



\hypertarget{dsa-winter-2023}{%
\subsection{1333203 DSA Winter 2023}\label{dsa-winter-2023}}

\hypertarget{q1a-define-linked-list.-list-different-types-of-linked-list.-03-marks}{%
\subsubsection{Q1a: Define linked list. List different types of linked
list. (03
marks)}\label{q1a-define-linked-list.-list-different-types-of-linked-list.-03-marks}}

\textbf{Ans 1a:} A linked list is a dynamic data structure consisting of
a sequence of elements, where each element (called a node) contains data
and a reference (or link) to the next element in the sequence. Unlike
arrays, linked lists do not store elements in contiguous memory
locations, allowing for efficient insertion and deletion operations.

Key characteristics of linked lists:

\begin{itemize}
\tightlist
\item
  Dynamic size: Can grow or shrink during program execution
\item
  Non-contiguous memory allocation: Elements can be stored anywhere in
  memory
\item
  Efficient insertion and deletion: O(1) time complexity for operations
  at the beginning or end
\end{itemize}

Different types of linked lists:

\begin{enumerate}
\def\labelenumi{\arabic{enumi}.}
\tightlist
\item
  \textbf{Singly Linked List}:

  \begin{itemize}
  \tightlist
  \item
    Each node contains data and a single reference to the next node
  \item
    Last node points to NULL, indicating the end of the list
  \end{itemize}
\item
  \textbf{Doubly Linked List}:

  \begin{itemize}
  \tightlist
  \item
    Each node contains data and two references: one to the next node and
    one to the previous node
  \item
    Allows traversal in both directions
  \end{itemize}
\item
  \textbf{Circular Linked List}:

  \begin{itemize}
  \tightlist
  \item
    Similar to singly linked list, but the last node points back to the
    first node
  \item
    Forms a closed loop
  \end{itemize}
\item
  \textbf{Circular Doubly Linked List}:

  \begin{itemize}
  \tightlist
  \item
    Combines features of doubly linked and circular linked lists
  \item
    Last node\textgujarati{’}s next pointer points to the first node,
    and first node\textgujarati{’}s previous pointer points to the last
    node
  \end{itemize}
\item
  \textbf{Header Linked List}:

  \begin{itemize}
  \tightlist
  \item
    Contains a special header node at the beginning
  \item
    Header node may store metadata about the list (e.g., size, pointers
    to first and last elements)
  \end{itemize}
\end{enumerate}

\hypertarget{uxaaauxab0uxab6uxaa8-1uxa85-uxab2uxaa8uxa95uxaa1-uxab2uxab8uxa9fuxaa8-uxab5uxaafuxa96uxaaf-uxa86uxaaa.-uxab5uxab5uxaa7-uxaaauxab0uxa95uxab0uxaa8-uxab2uxaa8uxa95uxaa1-uxab2uxab8uxa9f-uxaa8-uxaafuxaa6-uxa86uxaaa.-uxae6uxae9-uxa97uxaa3}{%
\subsubsection{\texorpdfstring{\textgujarati{પ્રશ્ન} 1\textgujarati{અ}:
\textgujarati{લીન્કડ} \textgujarati{લીસ્ટની} \textgujarati{વ્યાખ્યા}
\textgujarati{આપો}. \textgujarati{વિવિધ} \textgujarati{પ્રકારના}
\textgujarati{લિન્ક્ડ} \textgujarati{લીસ્ટ} \textgujarati{ની}
\textgujarati{યાદી} \textgujarati{આપો}. (\textgujarati{૦૩}
\textgujarati{ગુણ})}{ 1:    .       . ( )}}\label{uxaaauxab0uxab6uxaa8-1uxa85-uxab2uxaa8uxa95uxaa1-uxab2uxab8uxa9fuxaa8-uxab5uxaafuxa96uxaaf-uxa86uxaaa.-uxab5uxab5uxaa7-uxaaauxab0uxa95uxab0uxaa8-uxab2uxaa8uxa95uxaa1-uxab2uxab8uxa9f-uxaa8-uxaafuxaa6-uxa86uxaaa.-uxae6uxae9-uxa97uxaa3}}

\textbf{\textgujarati{જવાબ} 1\textgujarati{અ}:} linked list
\textgujarati{એ} \textgujarati{એક} \textgujarati{ડાયનેમિક}
\textgujarati{ડેટા} \textgujarati{સ્ટ્રક્ચર} \textgujarati{છે}
\textgujarati{જેમાં} \textgujarati{એલિમેન્ટ્સનો} \textgujarati{ક્રમ}
\textgujarati{હોય} \textgujarati{છે}, \textgujarati{જ્યાં}
\textgujarati{દરેક} \textgujarati{એલિમેન્ટ} (\textgujarati{જેને} node
\textgujarati{કહેવાય} \textgujarati{છે}) \textgujarati{ડેટા}
\textgujarati{અને} \textgujarati{ક્રમમાં} \textgujarati{આગળના}
\textgujarati{એલિમેન્ટનો} \textgujarati{સંદર્ભ} (\textgujarati{અથવા} link)
\textgujarati{ધરાવે} \textgujarati{છે}. \textgujarati{એરેઝથી}
\textgujarati{વિપરીત}, linked lists \textgujarati{એલિમેન્ટ્સને}
\textgujarati{સતત} \textgujarati{મેમરી} \textgujarati{સ્થાનોમાં}
\textgujarati{સંગ્રહિત} \textgujarati{કરતા} \textgujarati{નથી},
\textgujarati{જે} insertion \textgujarati{અને} deletion
\textgujarati{ઓપરેશન્સને} \textgujarati{કાર્યક્ષમ} \textgujarati{બનાવે}
\textgujarati{છે}.

linked lists \textgujarati{ની} \textgujarati{મુખ્ય}
\textgujarati{લાક્ષણિકતાઓ}:

\begin{itemize}
\tightlist
\item
  \textgujarati{ડાયનેમિક} \textgujarati{કદ}: \textgujarati{પ્રોગ્રામ}
  \textgujarati{એક્ઝીક્યુશન} \textgujarati{દરમિયાન} \textgujarati{વધી}
  \textgujarati{અથવા} \textgujarati{ઘટી} \textgujarati{શકે}
  \textgujarati{છે}
\item
  \textgujarati{નોન}-\textgujarati{કન્ટીગ્યુઅસ} \textgujarati{મેમરી}
  \textgujarati{એલોકેશન}: \textgujarati{એલિમેન્ટ્સ} \textgujarati{મેમરીમાં}
  \textgujarati{ગમે} \textgujarati{ત્યાં} \textgujarati{સ્ટોર}
  \textgujarati{કરી} \textgujarati{શકાય} \textgujarati{છે}
\item
  \textgujarati{કાર્યક્ષમ} insertion \textgujarati{અને} deletion:
  \textgujarati{શરૂઆત} \textgujarati{અથવા} \textgujarati{અંતમાં}
  \textgujarati{ઓપરેશન્સ} \textgujarati{માટે} O(1) \textgujarati{સમય}
  \textgujarati{જટિલતા}
\end{itemize}

\textgujarati{વિવિધ} \textgujarati{પ્રકારના} linked lists:

\begin{enumerate}
\def\labelenumi{\arabic{enumi}.}
\tightlist
\item
  \textbf{Singly Linked List}:

  \begin{itemize}
  \tightlist
  \item
    \textgujarati{દરેક} node \textgujarati{ડેટા} \textgujarati{અને}
    \textgujarati{આગળના} node \textgujarati{નો} \textgujarati{એક}
    \textgujarati{સિંગલ} \textgujarati{સંદર્ભ} \textgujarati{ધરાવે}
    \textgujarati{છે}
  \item
    \textgujarati{છેલ્લું} node NULL \textgujarati{તરફ} \textgujarati{પોઈન્ટ}
    \textgujarati{કરે} \textgujarati{છે}, \textgujarati{જે}
    \textgujarati{લિસ્ટના} \textgujarati{અંતને} \textgujarati{સૂચવે}
    \textgujarati{છે}
  \end{itemize}
\item
  \textbf{Doubly Linked List}:

  \begin{itemize}
  \tightlist
  \item
    \textgujarati{દરેક} node \textgujarati{ડેટા} \textgujarati{અને}
    \textgujarati{બે} \textgujarati{સંદર્ભો} \textgujarati{ધરાવે}
    \textgujarati{છે}: \textgujarati{એક} \textgujarati{આગળના} node
    \textgujarati{માટે} \textgujarati{અને} \textgujarati{એક}
    \textgujarati{પાછલા} node \textgujarati{માટે}
  \item
    \textgujarati{બંને} \textgujarati{દિશાઓમાં} \textgujarati{ટ્રાવર્સલની}
    \textgujarati{મંજૂરી} \textgujarati{આપે} \textgujarati{છે}
  \end{itemize}
\item
  \textbf{Circular Linked List}:

  \begin{itemize}
  \tightlist
  \item
    Singly Linked List \textgujarati{જેવું} \textgujarati{જ},
    \textgujarati{પરંતુ} \textgujarati{છેલ્લું} node \textgujarati{પ્રથમ} node
    \textgujarati{તરફ} \textgujarati{પાછું} \textgujarati{પોઈન્ટ}
    \textgujarati{કરે} \textgujarati{છે}
  \item
    \textgujarati{બંધ} \textgujarati{લૂપ} \textgujarati{બનાવે}
    \textgujarati{છે}
  \end{itemize}
\item
  \textbf{Circular Doubly Linked List}:

  \begin{itemize}
  \tightlist
  \item
    Doubly Linked \textgujarati{અને} Circular Linked Lists
    \textgujarati{ની} \textgujarati{વિશેષતાઓને} \textgujarati{જોડે}
    \textgujarati{છે}
  \item
    \textgujarati{છેલ્લા} node \textgujarati{નો} next pointer
    \textgujarati{પ્રથમ} node \textgujarati{તરફ} \textgujarati{પોઈન્ટ}
    \textgujarati{કરે} \textgujarati{છે}, \textgujarati{અને}
    \textgujarati{પ્રથમ} node \textgujarati{નો} previous pointer
    \textgujarati{છેલ્લા} node \textgujarati{તરફ} \textgujarati{પોઈન્ટ}
    \textgujarati{કરે} \textgujarati{છે}
  \end{itemize}
\item
  \textbf{Header Linked List}:

  \begin{itemize}
  \tightlist
  \item
    \textgujarati{શરૂઆતમાં} \textgujarati{એક} \textgujarati{વિશેષ} header
    node \textgujarati{ધરાવે} \textgujarati{છે}
  \item
    Header node \textgujarati{લિસ્ટ} \textgujarati{વિશેના}
    \textgujarati{મેટાડેટા} \textgujarati{સ્ટોર} \textgujarati{કરી}
    \textgujarati{શકે} \textgujarati{છે}
    (\textgujarati{દા}.\textgujarati{ત}., \textgujarati{કદ},
    \textgujarati{પ્રથમ} \textgujarati{અને} \textgujarati{છેલ્લા}
    \textgujarati{એલિમેન્ટ્સના} pointers)
  \end{itemize}
\end{enumerate}

\end{document}
