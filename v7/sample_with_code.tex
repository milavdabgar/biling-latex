\documentclass[12pt]{article}
\usepackage{fontspec}
\usepackage{polyglossia}
\usepackage{xparse}
\usepackage{enumitem}
\usepackage{fancyvrb}
\usepackage{listings}
\usepackage{xcolor}

\setmainlanguage{english}
\setotherlanguage{gujarati}

% Set fonts compatible with Claude's output
\setmainfont{Noto Sans}
\newfontfamily\gujaratifont{Noto Sans Gujarati}[
    Script=Gujarati,
    Language=Gujarati
]

% Define a command to automatically switch between English and Gujarati
\ExplSyntaxOn
\NewDocumentCommand{\autoswitch}{+m}
 {
  \tl_set:Nn \l_tmpa_tl { #1 }
  \regex_replace_all:nnN
   { ([ઁ-૱]+) }
   { \c{customgujarati}\cB\{\1\cE\} }
   \l_tmpa_tl
  \tl_use:N \l_tmpa_tl
 }
\ExplSyntaxOff

\newcommand{\customgujarati}[1]{{\gujaratifont #1}}

% Define tightlist command
\providecommand{\tightlist}{%
  \setlength{\itemsep}{0pt}\setlength{\parskip}{0pt}}

% Configure Verbatim for code highlighting
\fvset{
  fontsize=\small,
  frame=single,
  numbers=left,
  numbersep=5pt,
  commandchars=\\\{\}
}

% Define colors for syntax highlighting
\definecolor{KeywordColor}{RGB}{0,112,32}
\definecolor{StringColor}{RGB}{64,112,160}
\definecolor{CommentColor}{RGB}{96,160,176}

% Configure listings for code highlighting
\lstset{
    basicstyle=\ttfamily\small,
    breaklines=true,
    columns=flexible,
    keepspaces=true,
    numbers=left,
    numberstyle=\tiny\color{gray},
    showstringspaces=false,
    stepnumber=1,
    rulecolor=\color{black},
    backgroundcolor=\color{white},
    keywordstyle=\color{KeywordColor}\bfseries,
    commentstyle=\color{CommentColor}\itshape,
    stringstyle=\color{StringColor},
    frame=single,
    framesep=3pt,
    xleftmargin=\fboxsep,
    xrightmargin=\fboxsep
}

% Define commands for syntax highlighting
\newcommand{\KeywordTok}[1]{\textcolor{KeywordColor}{\textbf{#1}}}
\newcommand{\StringTok}[1]{\textcolor{StringColor}{#1}}
\newcommand{\CommentTok}[1]{\textcolor{CommentColor}{\textit{#1}}}

% Define the Shaded environment
\newenvironment{Shaded}{}{}

% Define the Highlighting environment
\DefineVerbatimEnvironment{Highlighting}{Verbatim}{commandchars=\\\{\}}

\begin{document}

\autoswitch{\subsubsection{Question 1(a): Define linear data structure
and give its examples. (03
marks)}\label{question-1a-define-linear-data-structure-and-give-its-examples.-03-marks}

\textbf{Ans 1(a):} A linear data structure is a type of data
organization where elements are arranged in a sequential manner, with
each element directly linked to its adjacent elements. Key
characteristics of linear data structures include:

\begin{itemize}
\tightlist
\item
  Elements are organized in a linear or sequential order.
\item
  Each element has a unique predecessor and successor, except for the
  first and last elements.
\item
  Data can be traversed in a single run, i.e., in one pass.
\end{itemize}

Examples of linear data structures:

\begin{enumerate}
\def\labelenumi{\arabic{enumi}.}
\tightlist
\item
  \textbf{Array}: A collection of elements stored in contiguous memory
  locations.
\item
  \textbf{Linked List}: A sequence of nodes where each node contains
  data and a reference to the next node.
\item
  \textbf{Stack}: Follows Last-In-First-Out (LIFO) principle for element
  access.
\item
  \textbf{Queue}: Follows First-In-First-Out (FIFO) principle for
  element access.
\end{enumerate}

\begin{lstlisting}[language=Python]
print("Hello")
\end{lstlisting}

\subsubsection{પ્રશ્ન 1(અ): રેખીય ડેટા સ્ટ્રક્ચર વ્યાખ્યાયિત કરો અને તેના ઉદાહરણો
આપો.
(૦૩)}\label{uxaaauxab0uxab6uxaa8-1uxa85-uxab0uxa96uxaaf-uxaa1uxa9f-uxab8uxa9fuxab0uxa95uxa9auxab0-uxab5uxaafuxa96uxaafuxaafuxaa4-uxa95uxab0-uxa85uxaa8-uxaa4uxaa8-uxa89uxaa6uxab9uxab0uxaa3-uxa86uxaaa.-uxae6uxae9}

\textbf{જવાબ 1(અ):} રેખીય ડેટા સ્ટ્રક્ચર એ ડેટા સંગઠનનો એક પ્રકાર છે જ્યાં elements
ક્રમિક રીતે ગોઠવાયેલા હોય છે, જેમાં દરેક element તેના આજુબાજુના elements સાથે સીધી
રીતે જોડાયેલો હોય છે. રેખીય ડેટા સ્ટ્રક્ચરની મુખ્ય લાક્ષણિકતાઓમાં સામેલ છે:

\begin{itemize}
\tightlist
\item
  Elements રેખીય અથવા ક્રમિક ક્રમમાં ગોઠવાયેલા હોય છે.
\item
  દરેક element ને અનન્ય પૂર્વગામી અને અનુગામી હોય છે, સિવાય કે પ્રથમ અને છેલ્લા
  elements.
\item
  ડેટાને એક જ run માં, એટલે કે એક પાસમાં traverse કરી શકાય છે.
\end{itemize}

રેખીય ડેટા સ્ટ્રક્ચરના ઉદાહરણો:

\begin{enumerate}
\def\labelenumi{\arabic{enumi}.}
\tightlist
\item
  \textbf{Array}: સળંગ મેમરી સ્થાનોમાં સંગ્રહિત elements નો સમૂહ.
\item
  \textbf{Linked List}: nodes નો ક્રમ જ્યાં દરેક node માં ડેટા અને આગળના node
  નો સંદર્ભ હોય છે.
\item
  \textbf{Stack}: Element access માટે Last-In-First-Out (LIFO) સિદ્ધાંતને
  અનુસરે છે.
\item
  \textbf{Queue}: Element access માટે First-In-First-Out (FIFO) સિદ્ધાંતને
  અનુસરે છે.
\end{enumerate}

\begin{lstlisting}[language=Python]
print("Hello")
\end{lstlisting}

\subsubsection{}\label{section}}

\end{document}
